% @(#) $Id$

\section{Reference}

\subsection{Software Requirements}

Mapyrus requires:
\begin{itemize}
\item
Java 2 Runtime Environment, Standard Edition, (J2RE) v1.4 or higher,
or Java 2 Software Developers Kit, Standard Edition (J2SDK) v1.4 or higher.
\item
The Ghostscript PostScript interpreter from 
\texttt{http://www.ghostscript.com}
if creation of PDF files is required.
\item
The \texttt{\$DISPLAY} environment variable set to an X-Windows display,
if running on Linux or a UNIX operating system (when a real X-Windows display
is not available, use the
\textit{X-Windows Virtual Framebuffer}, part of \textit{XFree86}
from
\texttt{http://www.xfree86.org}).
\item
The \textit{Java Topology Suite} from \texttt{http://www.vividsolutions.com},
if geometric functions are required
(see Table \ref{functions} on page \pageref{functions}).
\item
The OGR library from
\texttt{http://openev.sourceforge.net}
if reading of GIS data formats other than ESRI Shape format
and relational database tables is required.
\end{itemize}

\subsection{Usage}

The Mapyrus software is contained in a single Java JAR file.
Start Mapyrus in a Java interpreter with the following command.

\vspace{10pt}
\texttt{java -classpath \textit{install-dir}/mapyrus.jar org.mapyrus.Mapyrus \textit{filename} \dots}
\vspace{10pt}

\textit{install-path} is the directory in which
Mapyrus is installed.  \textit{filename} is the name of a file
or a URL for Mapyrus to read from.  If \textit{filename} is \texttt{-}
then standard input is read.  If several filenames and URLs are
given then they are read in turn.

Variables passed to Mapyrus using the Java \texttt{-D} option
are available in the Mapyrus interpreter.  The
\textit{Java Topology Suite} JAR file and other JAR files
to be used in combination with Mapyrus are included in the
\texttt{-classpath} option.

\begin{alltt}
\texttt{java -D\textit{variable}=\textit{value} \dots\ \textbackslash
  -classpath \textit{install-dir}/mapyrus.jar:\textit{jts-dir}/jts-1.4.0.jar:\textit{other-jarfile} \textbackslash
  org.mapyrus.Mapyrus \textit{filename}}
\end{alltt}

Mapyrus runs as an HTTP Server when started with the
\texttt{-s} option.

\begin{alltt}
\texttt{java -classpath \textit{install-dir}/mapyrus.jar:\textit{jarfile} \textbackslash
  org.mapyrus.Mapyrus -s \textit{port} \textit{filename} \dots}
\end{alltt}

\label{morememory}
Use the
\texttt{-Xmx} Java option
to make more memory available when running Mapyrus.
To increase available memory to 256Mb, use the following command:

\begin{alltt}
\texttt{java -Xmx256m -classpath \textit{install-dir}/mapyrus.jar org.mapyrus.Mapyrus \textit{filename} \dots}
\end{alltt}

\subsubsection{Startup Configuration}

The variables available for configuring Mapyrus at startup are
shown in Table \ref{startupvariables}.

\begin{longtable}{|l|p{7cm}|}
\hline
\label{startupvariables}
Variable & Description \\
\hline
\hline
\endfirsthead
\hline
\caption{Startup Variables} \\
\endfoot

\hline
Variable & Description \\
\hline
\hline
\endhead

\texttt{Mapyrus.rgb.file=\textit{filename}} &
Defines an X Windows color names file containing additional color names
for the
\texttt{color} command.  Default value is
\texttt{/usr/lib/X11/rgb.txt} \\

\hline

\texttt{java.util.logging.config.file=\textit{filename}} &
Defines file controlling logging for the
HTTP Server option.
See the Java Logging Overview
\footnote{Available from 
\texttt{http://java.sun.com/j2se/1.4/docs/guide/util/logging/overview.html}}
and Java java.util.LogManager API documentation
\footnote{Available from
\texttt{http://java.sun.com/j2se/1.4.2/docs/api/java/util/logging/LogManager.html}}
for format of the file defining logging properties.
Logger for Mapyrus is named \texttt{org.mapyrus.Mapyrus}.
Default logging writes only errors to error output. \\

\hline

\texttt{jdbc.drivers=\textit{class}} &
Defines class containing JDBC 1.0 (or higher) driver to load at startup.
A JDBC driver is required for connecting to a relational database and
is provided as part of a relational database.
See the Java JDBC DriverManager API documentation
\footnote{Available from
\texttt{http://java.sun.com/j2se/1.4.2/docs/api/java/sql/DriverManager.html}}
for details.
The JAR file containing the class must be included in the \texttt{-classpath}
option when starting Mapyrus. \\

\hline
\end{longtable}

\subsection{Language}

Mapyrus interprets commands read from one or more plain text files.
Each command begins on a separate line.

Any part of a line following a hash (\texttt{\#}) character that is not part of
a literal string is interpreted as a comment and is ignored.  Leading and
trailing spaces or tabs on a line are also ignored.  A backslash
(\texttt{\textbackslash}) character at the end of a line is interpreted as a
line continuation and the line and next line are joined into a single line.

A line beginning with the word \texttt{include}, followed by a filename
or URL includes commands from another file.

Each command is passed zero or more arguments separated by commas.
An argument is a number, a string literal in single quotes (\texttt{\'})
or double quotes (\texttt{\"}), a variable name, an array, an
array element or an expression.

The character sequence (\texttt{\textbackslash{}\textit{nnn}}) in
a string literal is interpreted as an octal character code (where
\texttt{\textit{nnn}} is one to three digits).

An expression contains arguments and operators and functions
on those arguments, like in BASIC, C, or Python.
Available operators are shown in Table \ref{operators}.
Available functions are shown in Table \ref{functions}.


\begin{table}[htb]
\begin{tabular}{|l|p{7cm}|}
\hline
Operator & Description \\
\hline
\hline

\texttt{(}, \texttt{)} & parentheses \\

\hline

\texttt{*}, \texttt{/}, \texttt{\%}, \texttt{x} &
numeric multiplication, numeric division, modulo (works
with non-integer values too), string repetition \\

\hline

\texttt{+}, \texttt{-}, \texttt{.} & numeric addition,
numeric subtraction, string concatenation \\

\hline

\texttt{<=}, \texttt{<}, \texttt{==},
\texttt{!=}, \texttt{>}, \texttt{>=},
\texttt{lt}, \texttt{le}, \texttt{eq},
\texttt{ne}, \texttt{gt}, \texttt{ge} &
numeric comparisons and string comparisons \\

\hline

\texttt{?:} & ternary conditional operator \\

\hline

\texttt{and},
\texttt{or},
\texttt{not} &
Logical and, or, not \\

\hline

\end{tabular}
\caption{Operators}
\label{operators}
\end{table}


\begin{longtable}{|p{5cm}|p{7cm}|}
\hline
\label{functions}
Function Name & Description \\
\hline
\hline
\endfirsthead
\hline
\caption{Functions} \\
\endfoot

\hline
Function Name & Description \\
\hline
\hline
\endhead

\texttt{abs(\textit{n})} &
Returns the absolute value of \textit{n}. \\

\hline

\texttt{buffer(\textit{g}, \textit{dist}, \textit{cap})} &
Returns a geometry containing a buffer calculated at
a distance \textit{dist} around the perimeter of geometry \textit{g}.
The value of \textit{cap} defines
the method of closing buffers at line endpoints, either
\texttt{butt}, \texttt{round} or \texttt{square}.
This function requires the \textit{Java Topology Suite}. \\

\hline

\texttt{ceil(\textit{n})} &
Returns the smallest integer value that is not less than \textit{n}. \\

\hline

\texttt{contains(\textit{g1}, \textit{x}, \textit{y})}
\texttt{contains(\textit{g1}, \textit{g2})} &
If point (\textit{x}, \textit{y}) or geometry \textit{g2}
is contained inside \textit{g1} then
1 is returned.  Otherwise 0 is returned.
This function requires the \textit{Java Topology Suite}. \\

\hline

\texttt{convexhull(\textit{g})} &
Returns a convex hull geometry that surrounds geometry \textit{g}. \\

\hline

\texttt{cos(\textit{n})} &
Returns the cosine of angle \textit{n}, given in degrees. \\

\hline

\texttt{difference(\textit{g1}, \textit{g2})} &
Returns a geometry containing the difference between
geometry \textit{g1} and geometry \textit{g2}.  That is, parts
of geometry \textit{g1} that are not part of geometry \textit{g2}.
This function requires the \textit{Java Topology Suite}. \\

\hline

\texttt{dir(\textit{p})} &
Returns an array of all filenames matching the wildcard
pattern \textit{p} containing asterisk (\texttt{*}) characters. \\

\hline

\texttt{floor(\textit{n})} &
Returns the largest integer value that is not larger than \textit{n}. \\

\hline

\texttt{interpolate(\textit{str}, \textit{n})} &
Returns value calculated from \textit{n} using linear interpolation.
\textit{str} contains list of numbers (given in increasing numeric
order) and corresponding values:
$n_{1}$ $v_{1}$ $n_{2}$ $v_{2}$ \ldots.
Result is found by finding range
$n_{i}$
to
$n_{i+1}$
containing \texttt{n} and
using linear interpolation to calculate a value between
$v_{i}$
and
$v_{i+1}$.
Each value $v_{i}$ is either a number or a named color.  \\

\hline

\texttt{intersection(\textit{g1}, \textit{g2})} &
Returns a geometry containing the intersection
of geometry \textit{g1} and geometry \textit{g2}.
This function requires the \textit{Java Topology Suite}. \\

\hline

\texttt{length(\textit{v})} &
If \textit{v} is an array, then the number of elements in the
array is returned.  Otherwise the string length of \textit{v} is returned. \\

\hline

\texttt{log10(\textit{n})} &
Returns the base 10 logarithm of \textit{n}. \\

\hline

\texttt{lower(\textit{str})} &
Returns \textit{str} converted to lower case. \\

\hline

\texttt{lpad(\textit{str}, \textit{len}, \textit{pad})}
\texttt{lpad(\textit{str}, \textit{len})} &
Returns string \textit{str} left padded to length \textit{len}
using characters from string \textit{pad}.
Spaces are used for padding if \textit{pad} is not given.
String is truncated on the left if longer than length \textit{len}. \\

\hline

\texttt{match(\textit{str}, \textit{regex})} &
Returns the index in the string \textit{str}, where the regular expression
\textit{regex} is first matched.  The index of the first character is 1.
If the regular expression does not match \textit{str}, then 0 is returned.
The \textit{Java API documentation}
\footnote{Available from
\texttt{http://java.sun.com/j2se/1.4.2/docs/api/java/util/regex/Pattern.html}}
describes the syntax of regular expressions. \\

\hline

\texttt{max(\textit{a}, \textit{b})} &
Returns the larger of values \textit{a} and \textit{b}. \\

\hline

\texttt{min(\textit{a}, \textit{b})} &
Returns the smaller of values \textit{a} and \textit{b}. \\

\hline

\texttt{overlaps(\textit{g1}, \textit{g2})} &
If geometry \textit{g1} and geometry \textit{g2}
are the same type
and overlap
then 1 is returned.  Otherwise 0 is returned.
This function requires the \textit{Java Topology Suite}. \\

\hline

\texttt{pow(\textit{a}, \textit{b})} &
Returns \textit{a} to the power \textit{b}. \\

\hline

\texttt{protected(\textit{x1}, \textit{y1}, \textit{x2}, \textit{y2})} &
Returns 1 if any part of the given rectangular region
has been protected using the \texttt{protect}
command.
The points
(\textit{x1}, \textit{y1}) and (\textit{x2}, \textit{y2}) define
any two opposite corners of the rectangle, in current world coordinate
units.
Otherwise 0 is returned. \\

\hline

\texttt{random(\textit{n})} &
Generates a random floating point number between 0 and \textit{n}. \\

\hline

\texttt{replace(\textit{str}, \textit{regex}, \textit{replacement})} &
Returns the string \textit{str}, with all occurrences of the regular
expression \textit{regex} replaced by \textit{replacement}. \\

\hline

\texttt{round(\textit{n})} &
Returns \textit{n} rounded to nearest whole number. \\

\hline

\texttt{rpad(\textit{str}, \textit{len}, \textit{pad})}
\texttt{rpad(\textit{str}, \textit{len})} &
Returns string \textit{str} right padded to length \textit{len}
using characters from string \textit{pad}.
Spaces are used for padding if \textit{pad} is not given.
String is truncated if longer than length \textit{len}. \\

\hline

\texttt{sin(\textit{n})} &
Returns the sine of angle \textit{n}, given in degrees. \\

\hline

\texttt{split(\textit{str}, \textit{regex})} &
Splits the string \textit{str} into an array of strings, delimited by the
regular expression \textit{regex}.
The array of split strings is returned,
with the first string having array index 1, the
second string having index 2, and so on. \\

\hline

\texttt{spool(\textit{filename})} &
Returns string containing contents of text file \textit{filename}.
If \textit{filename} has suffix
\texttt{.gz} or \texttt{.zip}
then it is automatically decompressed as it is read.  \\

\hline

\texttt{sqrt(\textit{n})} &
Returns square root of \textit{n}. \\

\hline

\texttt{stringheight(\textit{str})} &
Returns the height in millimeters of the string \textit{str}
if it were displayed using the \texttt{label} command.
For strings containing several lines, the total height of all
lines is returned.  \\

\hline

\texttt{stringwidth(\textit{str})} &
Returns the width in millimeters of the string \textit{str}
if it were displayed using the \texttt{label} command.
For strings containing several lines, the width of the longest
line is returned.  \\

\hline

\texttt{substr(\textit{str}, \textit{offset}, \textit{n})}
\texttt{substr(\textit{str}, \textit{offset})} &
Returns a substring of the string \textit{str}, beginning at the
character with index \textit{offset} that is \textit{n} characters long,
or all characters from index \textit{offset} if
\textit{n} is not given.
The first character in \textit{str} has an index of 1. \\

\hline

\texttt{tan(\textit{n})} &
Returns the trigonometric tangent of angle \textit{n}, given in degrees. \\

\hline

\texttt{tempname(\textit{suffix})} &
Returns a unique temporary filename with given file suffix,
for use when running as an HTTP server.
Temporary files returned by this function
are automatically deleted after 5 minutes. \\

\hline
\texttt{trim(\textit{str})} &
Returns string \textit{str} with whitespace trimmed from start and end. \\

\hline

\texttt{union(\textit{g1}, \textit{g2})} &
Returns a geometry containing the union of 
of geometry \textit{g1} and geometry \textit{g2}.
This function requires the \textit{Java Topology Suite}. \\

\hline

\texttt{upper(\textit{str})} &
Returns \textit{str} converted to upper case. \\

\hline

\texttt{wordwrap(\textit{str}, \textit{width})} &
Returns \textit{str} broken into several lines for
use in a \texttt{label} command.
Each line will not be
longer than \textit{width} millimeters wide. \\

\hline
\end{longtable}

An argument or expression is assigned to a named variable using the
\texttt{let} command and an equals sign (\texttt{=}).  A variable name begins
with a letter or dollar sign (\texttt{\$}) and contains only letters, numbers,
dots (\texttt{.}), underbars (\texttt{\_}) and colons (\texttt{:}).
Variable names are
case-sensitive.  Variables and array elements that are accessed without having
been defined have a value of zero, or an empty string.

Variables accepted as parameters to a procedure, or declared in the procedure
with the local command are local to that procedure and not visible outside the
procedure.  All other variables are global.

Array elements are accessed by giving the index between
square brackets (\texttt{[} and \texttt{]}).
The index is any string or number value.
Multi-dimension arrays are available by using
indexes named \textit{index1}\texttt{."}\textit{c}\texttt{".}\textit{index2},
where \textit{c} is any character that never appears in an index.

The \texttt{if}, \texttt{repeat}, \texttt{while} and \texttt{for}
flow control structures
found in other languages are available:

\begin{alltt}
if \textit{condition} then
  \textit{then-commands} ...
else
  \textit{else-commands} ...
endif
\end{alltt}

Executes \textit{then-commands} if \textit{condition} evaluates to
a non-zero value, otherwise \textit{else-commands} are executed.
The \texttt{else} part is optional.
Compound tests are built using the \texttt{elif} keyword:

\begin{alltt}
if \textit{condition} then
  \textit{commands} \dots
elif \textit{condition} then
  \textit{commands} \dots
endif
\end{alltt}

The \texttt{repeat} keyword defines a loop in which
\textit{commands} will be executed \textit{count}
times:

\begin{alltt}
repeat \textit{count} do
  \textit{commands} \dots
done
\end{alltt}

The \texttt{while} keyword defines a loop in which
\textit{commands} will be executed for as long
as \textit{condition} continues to evaluate to a non-zero value:

\begin{alltt}
while \textit{condition} do
  \textit{commands} \dots
done
\end{alltt}

The \texttt{for} \dots \texttt{in} keywords define a loop in which
each element of \textit{array} is assigned to variable \textit{var}
and \textit{commands} are executed.
Elements in \textit{array} are accessed in numerical index order if
indexes are numeric, otherwise in alphabetical index order:

\begin{alltt}
for \textit{var} in \textit{array} do
  \textit{commands} \dots
done
\end{alltt}

Procedures group frequently used commands together, save
graphics state when they begin and restore it when they end,
isolating the calling procedure from any changes:

\begin{alltt}
begin \textit{name} [\textit{arg1}, \dots]
  \textit{commands} \dots
end
\end{alltt}

A procedure is defined to take a fixed number of arguments.
All procedure names are global and following the same naming
rules as variables.
Procedure definitions within a procedure are not allowed.
The \texttt{return} keyword returns a
from a procedure to the calling procedure.

A procedure is called from any place where a command is accepted
and the number of arguments passed must match the number 
in the procedure definition.

Before commands in the procedure are executed
the graphics state is saved.
\label{graphicsstate}
The graphics state is restored when the procedure finishes.
The graphics state contains:

\begin{enumerate}
\item
The path defined with
\texttt{move},
\texttt{draw},
\texttt{box},
\texttt{arc},
\texttt{circle},
\texttt{ellipse},
\texttt{hexagon},
\texttt{star},
\texttt{triangle},
\texttt{wedge}
and
\texttt{addpath}
commands.

\item
The clip path defined with
\texttt{clip} commands.

\item
Current drawing settings set by
\texttt{color},
\texttt{linestyle},
\texttt{font} and \texttt{justify}
commands.

\item
Transformations set by \texttt{rotate} and \texttt{scale} commands.
\end{enumerate}

Any new page created in a procedure with a
\texttt{newpage}
command is completed when the procedure finishes and output
returns to the page being created before the procedure was called.

Any dataset being read is global and protected regions are global.
They are not saved and not restored.

Any world coordinate system set with the \texttt{worlds} command is cleared
before the commands in the procedure are executed.  This enables the calling
procedure to work in a world coordinate system and the called procedure to draw
at these world coordinate positions using measurements in millimeters.

If the current path contains only \texttt{move} points and no straight line or
arc segments when a procedure is called then the procedure is called
repeatedly, one time for each move point, with the origin (0, 0) translated to
the \texttt{move} point each time and the path reset to empty.  Therefore,
coordinates in the called procedure are relative to the move point.  This
enables drawing commands in the called procedure to be given in millimeters,
relative to each \texttt{move} point.

\subsection{Internal Variables}

All Java standard system properties (\texttt{os.arch}, \texttt{user.dir}, etc.)
are defined as variables in Mapyrus.

Mapyrus maintains the internal variables shown in Table \ref{internalvariables}.

\begin{longtable}{|p{6cm}|p{7cm}|}
\hline
\label{internalvariables}
Variable Name & Description \\
\hline
\hline
\endfirsthead
\hline
\caption{Internal Variables} \\
\endfoot

\hline
Variable Name & Description \\
\hline
\hline
\endhead

\texttt{Mapyrus.dataset.fieldnames} &
An array containing the names of fields being read from
the current dataset.  The first fieldname has array index 1. \\

\hline

\texttt{Mapyrus.dataset.projection} &
A description of the projection (coordinate system) in which coordinates
of the current dataset are stored.
Projection descriptions are not standardised between dataset formats.
Different dataset formats will return different descriptions for the same
projection. \\

\hline

\texttt{Mapyrus.dataset.min.x},
\texttt{Mapyrus.dataset.min.y},
\texttt{Mapyrus.dataset.max.x},
\texttt{Mapyrus.dataset.max.y} &
The bounding rectangle of all data in the current dataset. \\

\hline

\texttt{Mapyrus.fetch.count} &
The number of records already fetched from the current dataset. \\

\hline

\texttt{Mapyrus.fetch.more} &
Flag value set to 1 if another record is available
for \texttt{fetch}
command, or 0 if no more records available. \\

\hline

\texttt{Mapyrus.filename} &
The name of the file or URL being interpreted. \\

\hline

\texttt{Mapyrus.freeMemory} &
The amount of free memory that Java has available, in bytes. \\

\hline

\texttt{Mapyrus.imagemap.x},
\texttt{Mapyrus.imagemap.y} &
The pixel position of the point clicked in an HTML imagemap and
passed to Mapyrus, for use when running as an HTTP server.
Both values are set to -1 if no imagemap point passed in current URL. \\

\hline

\texttt{Mapyrus.key.count} &
The number of legend entries defined with
\texttt{key} commands that have not yet
been displayed with a
\texttt{legend} command. \\

\hline

\texttt{Mapyrus.key.next} &
The name of the of the next procedure to be displayed by the
\texttt{legend} command. \\

\hline

\texttt{Mapyrus.page.format},
\texttt{Mapyrus.page.height},
\texttt{Mapyrus.page.width},
\texttt{Mapyrus.page.resolution.dpi},
\texttt{Mapyrus.page.resolution.mm} &
The file format, page height, page width
and resolution that were passed to the
\texttt{newpage} command.  File format is in lowercase.
Height and width are in millimeters.  Resolution is available
as either a dots-per-inch value, or a distance in millimeters between
dots. \\

\hline

\texttt{Mapyrus.path.length},
\texttt{Mapyrus.path.area},
\texttt{Mapyrus.path.min.x},
\texttt{Mapyrus.path.min.y},
\texttt{Mapyrus.path.max.x},
\texttt{Mapyrus.path.max.y},
\texttt{Mapyrus.path.width},
\texttt{Mapyrus.path.height} &
The length of the current path on the page measured in millimeters,
the area of the current path measured in square millimeters
and the bounding rectangle of the current path. \\

\hline

\texttt{Mapyrus.rotation} &
The current rotation angle in degrees set by
\texttt{rotate} command.
Returned value is normalised to fall in the
range -180 to +180 degrees. \\

\hline

\texttt{Mapyrus.scale} &
The current scale factor set by \texttt{scale} command. \\

\hline

\texttt{Mapyrus.screen.height},
\texttt{Mapyrus.screen.width},
\texttt{Mapyrus.screen.resolution.dpi},
\texttt{Mapyrus.screen.resolution.mm} &
The height, width and resolution of the screen in which Mapyrus
is running.
Height and width are in millimeters.  Resolution is available
as either a dots-per-inch value, or a distance in millimeters between
dots. \\

\hline

\texttt{Mapyrus.time.day},
\texttt{Mapyrus.time.month},
\texttt{Mapyrus.time.year},
\texttt{Mapyrus.time.hour},
\texttt{Mapyrus.time.minute},
\texttt{Mapyrus.time.second},
\texttt{Mapyrus.time.day.of.week},
\texttt{Mapyrus.time.day.name},
\texttt{Mapyrus.time.month.name},
\texttt{Mapyrus.time.week.of.year},
\texttt{Mapyrus.time.stamp}
&
Components of the current date and time.
Day of week has value 1 for Monday through to 7 for Sunday.

\\

\hline

\texttt{Mapyrus.timer} &
The elapsed processing time, measured in seconds. \\

\hline

\texttt{Mapyrus.totalMemory} &
The total amount of memory available to Java, in bytes. \\

\hline

\texttt{Mapyrus.version} &
The version of the software. \\

\hline

\texttt{Mapyrus.worlds.min.x},
\texttt{Mapyrus.worlds.min.y},
\texttt{Mapyrus.worlds.max.x},
\texttt{Mapyrus.worlds.max.y},
\texttt{Mapyrus.worlds.width},
\texttt{Mapyrus.worlds.height} &
The bounding rectangle of world coordinates set with the 
\texttt{worlds} command. \\

\hline

\texttt{Mapyrus.worlds.scale} &
The real-world scale factor, determined by
dividing of the X axis world coordinate range
by the page width. \\

\end{longtable}

\subsection{Commands}

Commands are listed alphabetically.  The arguments required for each command
are given.  Some commands accept arguments in several ways.  For these
commands, each combination of arguments is given.

\subsubsection{addpath}

\begin{alltt}
addpath \textit{geometry-field} [, \textit{geometry-field} ...]
\end{alltt}

Adds geometry in each \textit{geometry-field} to current path.
A \textit{geometry-field} is geometry fetched from a dataset
with a \texttt{fetch} command or a string containing an
OGC WKT geometry.

Coordinates are transformed through any
transformation set with a \texttt{worlds} command,
then scaled and rotated by \texttt{scale}
and \texttt{rotate} values.

\subsubsection{arc}

\begin{alltt}
arc \textit{direction}, \textit{xCenter}, \textit{yCenter}, \textit{xEnd}, \textit{yEnd}
\end{alltt}

Adds a circular arc to the current path.  The arc begins at
the last point added to the path and ends at (\textit{xEnd}, \textit{yEnd})
with center at (\textit{xCenter}, \textit{yCenter}).
If \textit{direction} is a positive number, the arc travels clockwise,
otherwise the arc travels in an anti-clockwise direction.
If the begin and end points are the same then the arc is a complete circle.
A straight line segment is first added to the path if
the distance from the beginning point to the center is different
to the distance from the center to the end point.

Points are transformed through any
transformation set with a \textit{worlds} command,
then scaled and rotated by \textit{scale}
and \textit{rotate} values.

\subsubsection{box}

\begin{alltt}
box \textit{x1}, \textit{y1}, \textit{x2}, \textit{y2}
\end{alltt}

Adds a rectangle to the current path.
The points
(\textit{x1}, \textit{y1}) and (\textit{x2}, \textit{y2}) define
any two opposite corners of the rectangle.

The four corner points of the box
are first transformed through any world coordinate
transformation set with a \texttt{worlds} command,
then scaled and rotated by \texttt{scale}
and \texttt{rotate} values.

\subsubsection{circle}

\begin{alltt}
circle \textit{xCenter}, \textit{yCenter}, \textit{radius}
\end{alltt}

Adds a circle to the current path, with center
point (\textit{xCenter}, \textit{yCenter}) and radius
\textit{radius}.

The center point and radius are transformed through any
transformation set with a \textit{worlds} command,
then scaled and rotated by \textit{scale}
and \textit{rotate} values.

\subsubsection{clearpath}

\begin{alltt}
clearpath
\end{alltt}

Removes all points from the current path.

\subsubsection{clip}

\begin{alltt}
clip \textit{side}
\end{alltt}

Sets a clip path to the area covered by the current path,
or excluding the area covered by the current path, depending
on the value \textit{side}.

If \textit{side} has value \texttt{inside} then
later drawing commands are limited to draw only inside the area
covered by current path.
If \textit{side} has value \texttt{outside} then
later drawing commands are limited to draw only outside the area
covered by current path.

If the path is clipped in a procedure, then the area remains
clipped until the procedure is complete.  Otherwise, the area
remains permanently clipped for the page.
When more than one path is clipped, drawing is limited to
areas that satisfy all clip paths.

The current path is not modified by this command.

\subsubsection{closepath}

\begin{alltt}
closepath
\end{alltt}

Closes the current path by adding a straight line segment
back to the last point added with a \texttt{move} command.

\subsubsection{color}

\begin{alltt}
color \textit{name} [, \textit{alpha}]
color "contrast" [, \textit{alpha}]
color "\#\textit{hexdigits}" [, \textit{alpha}]
color "0x\textit{hexdigits}" [, \textit{alpha}]
color "rgb", \textit{red}, \textit{green}, \textit{blue} [, \textit{alpha}]
color "hsb", \textit{hue}, \textit{saturation}, \textit{brightness} [, \textit{alpha}]
\end{alltt}

Sets color for drawing.  Around 150 commonly used color names are defined,
additional color names are defined in a file given as a startup variable (see
Table \ref{startupvariables} on page \pageref{startupvariables}).
Color names are case-insensitive.

If color name is \texttt{contrast} then color is set to either black or white,
whichever constrasts more with the current color.

\textit{hexdigits} is a 6 digit hexadecimal
value defining RGB values, as used in HTML pages.

\textit{red}, \textit{green} and \textit{blue} values for RGB colors and
\textit{hue}, \textit{saturation} and \textit{brightness} values for
Hue-saturation-brightness (HSB) colors are given as intensities in the range
0-1.

The alpha value is optional and defines transparency as a value in the range
0-1.  An alpha value of 1 is completely opaque and the color overwrites
underlying colors.  An alpha value of 0 is completely transparent and the color
is not visible.  Intermediate values are partially transparent and the color is
blended with colors of underlying shapes on the page.

Colors are opaque if an alpha value is not given.

Transparent colors are only available for JPEG, PNG
and Encapsulated PostScript format image (\texttt{epsimage}) output.

The PostScript language does not contain any functions for setting
transparency.
All colors in PostScript and Encapsulated PostScript
files will be opaque.

The spelling \texttt{colour} is also accepted for this command.

\subsubsection{dataset}

\begin{alltt}
dataset \textit{type}, \textit{name}, \textit{extras}
\end{alltt}

Defines a dataset to read from.  A dataset contains geographic data,
geometry, attributes, a lookup table, data to write to standard
output, or a combination of these.

\textit{dataset} is the filename of the dataset to read.
\textit{type} is the type of dataset and
\textit{extras} defines further options for accessing the dataset, given
as \textit{variable=value} values, separated by whitespace.
Dataset types and options are shown in Table \ref{datasettypes}.


\begin{longtable}{|p{3cm}|p{10cm}|}
\hline
\label{datasettypes}
Dataset Type & Description and Extras \\
\hline
\hline
\endfirsthead
\hline
\caption{Dataset Types} \\
\endfoot

\hline
Dataset Type & Description and Extras \\
\hline
\hline
\endhead

\texttt{grass} &
Reads from GRASS site list file with filename \textit{name}.
A GRASS site list is stored in a \texttt{site\_lists}
directory in a GRASS mapset.
The geometry for each site point is assigned to a variable named
\texttt{GEOMETRY}, attribute field values are assigned to variables
\texttt{\$1}, \texttt{\$2}, \texttt{\$3}, \dots
If the site file contains three dimensional data then the
third dimension is assigned to a variable named \texttt{Z}.  \\

\hline

\texttt{jdbc} &
Accesses data held in a relational database with
an SQL \texttt{select} statement via JDBC.
\textit{name} contains the SQL query to execute.
For each fetched record, field values are assigned to variables
with the name of the fields.
Field values that are NULL are converted to either an empty string,
or numeric zero, depending on their type.
Binary and blob fields are interpreted as OGC WKB geometry values.

Some databases convert all field names to upper case, or to lowercase.
Use a field name alias for fields that are the result of an expression.

\vspace{10pt}
Extras:

\texttt{driver=\textit{string}}

The name of the Java class containing a JDBC 1.0 (or higher)
driver for connecting to the database.
This class name is required if not given in the startup variable
\texttt{jdbc.drivers} (see Table \ref{startupvariables}
on page \pageref{startupvariables}).
The JAR file containing the class must be included in the \texttt{-classpath}
option when starting Mapyrus.

\vspace{10pt}
\texttt{url=\textit{string}}

URL containing the database name, host and other information for identifying
the database to connect to.
The format of this string is database dependent.
The database remains connected after use and the connection is reused in later
\texttt{dataset} commands with the same \texttt{url} value.
When Mapyrus is run using the HTTP Server option,
requests using with the same \texttt{url} value share a
single database connection.

\vspace{10pt}
\texttt{user=\textit{string}}

Username for connecting to the database.

\vspace{10pt}
\texttt{password=\textit{string}}

Password for connecting to the database.

\vspace{10pt}
Other values are set as properties for the JDBC driver. \\

\hline

\texttt{ogrinfo} &
Reads the output of the \texttt{ogrinfo} program, part of
the freely-available OGR library.
The filename \textit{name} contains either,
\begin{enumerate}
\item
An \texttt{ogrinfo} program command followed by a pipe (\texttt{|}) character.
Mapyrus runs this command and reads the output.  Use \texttt{ogrinfo}
command options to limit the data to an area of interest, or to data
matching an SQL-like where clause.
\item
(\texttt{-}) to read standard input to Mapyrus.
\item
The name of a file containing \texttt{ogrinfo} output.
\end{enumerate}

The geometry for each fetched record is assigned to a variable named
\texttt{GEOMETRY}, attribute field values are assigned to
variables with attribute field names. \\

\hline

\texttt{shapefile} &
Reads from ESRI Shape format file with filename \textit{name}.
The geometry for each fetched record is assigned to a variable named
\texttt{GEOMETRY}, attribute field values are assigned to
variables with attribute field names.

\vspace{10pt}
Extras:

\texttt{dbffields=\textit{field1},\textit{field2},...}

Comma-separated list of
attribute fields to read from the DBF database file accompanying the
Shape file.  By default, all fields are read.  Reading fewer attribute
fields improves performance.

\vspace{10pt}
\texttt{xmin=\textit{x1}},
\texttt{ymin=\textit{y1}},
\texttt{xmax=\textit{x2}},
\texttt{ymax=\textit{y2}}

Bounding rectangle of data to fetch.  Data outside this rectangle is
not fetched.  Setting bounding rectangle to same values as world
coordinate values in \texttt{worlds} command improves performance. \\

\hline

\texttt{textfile} &
Reads from delimited text file \textit{name}, with one
record per line.  Fields in fetched record
are assigned to variables
\texttt{\$1}, \texttt{\$2}, \texttt{\$3}, \dots
and the whole record is assigned to variable
\texttt{\$0}.
If \textit{name} is \texttt{-} then standard input is read.
If \textit{name} has suffix \texttt{.gz} or \texttt{.zip}
then it is automatically decompressed as it is read.

\vspace{10pt}
Extras:

\texttt{comment=\textit{string}}

Character string at start of a line marking a comment line that
is to be ignored.  Default value is a hash character (\texttt{\#}).

\vspace{10pt}
\texttt{delimeters=\textit{string}}

Characters separating fields in the text file.  Default value
is all whitespace characters. \\

\end{longtable}

\subsubsection{draw}

\begin{alltt}
draw \textit{x}, \textit{y}, ...
\end{alltt}

Adds one or more straight line segments to the current path.
A straight line segment is added from the previously defined point
to (\textit{x}, \textit{y}) and then to each further point given.
Points are first transformed through any world coordinate
transformation set with a \texttt{worlds} command
then scaled and rotated by \texttt{scale}
and \texttt{rotate} values.

\subsubsection{ellipse}

\begin{alltt}
ellipse \textit{xCenter}, \textit{yCenter}, \textit{xRadius}, \textit{yRadius}
\end{alltt}

Adds an ellipse to the current path, with center
point (\textit{xCenter}, \textit{yCenter}).  The radius of the ellipse
in the horizontal direction is \textit{xRadius} and in the vertical
direction \textit{yRadius}.

The center point and radius values are transformed through any
transformation set with a \textit{worlds} command,
then scaled and rotated by \textit{scale}
and \textit{rotate} values.

\subsubsection{eval}

\begin{alltt}
eval \textit{command}
\end{alltt}

Evaluates any variables in \textit{command} and then
runs the result as a new command.
This command is identical to the \texttt{eval} command found in
UNIX scripting and Perl and
enables commands to be built and executed while Mapyrus runs.

\subsubsection{fetch}

\begin{alltt}
fetch
\end{alltt}

Fetches next record from current dataset.
For each field in the record, a variable is defined with the name
of the field and the value of the field for the next record.
Before fetching a record, check the variable
\texttt{Mapyrus.fetch.more}
to ensure that another record is available from the dataset.

\subsubsection{fill}

\begin{alltt}
fill
\end{alltt}

Flood fills the current path with the current color.
The winding rule is used for determining the inside and outside
regions of polygons containing islands.
The current path is not modified by this command.

\subsubsection{flowlabel}

\begin{alltt}
flowlabel \textit{spacing}, \textit{offset}, \textit{string} [, \textit{string} ...]
\end{alltt}

Draws a label following the current path,
using the font set with the \texttt{font} command.
\textit{string} values are
separated by spaces.
\textit{offset} is the distance along the path at which to begin
the label, given in millimeters.
\textit{spacing} is the spacing distance
between each letter, given in millimeters.

If the label would appear upside down on the page, then it is rotated
180 degrees so it appears the right way up and is readable.

\subsubsection{font}

\begin{alltt}
font \textit{name}, \textit{size} [, \textit{extras} ...]
\end{alltt}

Sets font for labelling with the \texttt{label} command.
Font \textit{name} and \textit{size} are the name and size in
millimeters of the font to use.

If a scale factor was set with the
\texttt{scale} command
then font size is scaled by this factor.

If a rotation was set with the
\texttt{rotate} command
then labels follow current rotation angle.
If no rotation is set then labels are displayed
horizontally.

Font \textit{name}
depends on the output format set with the
\texttt{newpage} command.
For PostScript output, \textit{name} is the name of a PostScript Type 1
font.
For output to an image format, \textit{name} is one of the Java Logical
font names (\texttt{Serif}, \texttt{SansSerif},
\texttt{Monospaced}, \texttt{Dialog}, or \texttt{DialogInput}) or a TrueType
font name.

Tutorial Sections \ref{psfonts} and \ref{ttffonts}
describe different font formats.

\textit{extras} defines further options for the font, given as
\textit{variable=value} values, separated by whitespace.
See Table \ref{fontextras}
for available options.

\begin{longtable}{|l|p{7cm}|}
\hline
\label{fontextras}
Extra & Description \\
\hline
\hline
\endfirsthead
\hline
\caption{Font Extras} \\
\endfoot

\hline
Extra & Description \\
\hline
\hline
\endhead

\texttt{outlinewidth=\textit{width}} &

Sets line width to use for drawing outline of each letter in label.
Only the outline of each letter is drawn, no part of the letter is filled. \\

\hline
\end{longtable}

\subsubsection{gradientfill}

\begin{alltt}
gradientfill \textit{color1}, \textit{color2}, \textit{color3}, \textit{color4} [, \textit{color5} \ldots]
\end{alltt}

Fills the current path with a gradient fill pattern.
Color names \textit{color1}, \textit{color2}, \textit{color3} and
\textit{color4} define the color for the lower-left corner,
lower-right corner, upper-left corner and upper-right corner of
the polygon.

If \textit{color5} is given then it defines an additional color at the
the center of the polygon.

Colors in the interior of the polygon fade from the color defined
in each corner to the colors in the other corners.

The winding rule is used for determining the inside and outside
regions of polygons containing islands.
The current path and current color are not modified by this command.

\subsubsection{guillotine}

\begin{alltt}
guillotine \textit{x1}, \textit{y1}, \textit{x2}, \textit{y2}
\end{alltt}

Cuts path against a rectangle.
Any part of the path inside or
on the boundary of the rectangle remains.
Any part of the path outside the rectangle is removed.
The points
(\textit{x1}, \textit{y1}) and (\textit{x2}, \textit{y2}) define
any two opposite corners of the rectangle to cut against.

The four corner points of the rectangle
are first transformed through any world coordinate
transformation set with a \texttt{worlds} command,
then scaled and rotated by \texttt{scale}
and \texttt{rotate} values.

The path is always cut against a rectangle
aligned with the X and Y axes of the page, regardless of any rotation angle.

\subsubsection{hexagon}

\begin{alltt}
hexagon \textit{xCenter}, \textit{yCenter}, \textit{radius}
\end{alltt}

Adds a hexagon shape to the current path, with center
point (\textit{xCenter}, \textit{yCenter}) and distance
\textit{radius}
from the center point to each vertex.

The center point and radius are transformed through any
transformation set with a \textit{worlds} command,
then scaled and rotated by \textit{scale}
and \textit{rotate} values.

\subsubsection{icon}

\begin{alltt}
icon \textit{filename}, [, \textit{size}]
icon "\textit{binarydigits}" [, \textit{size}]
icon "0x\textit{hexdigits}" [, \textit{size}]
\end{alltt}

Displays an image icon at each \texttt{move} point in
the current path.  The icon is centered at each point.

\textit{filename} is a the name of a file or URL containing the icon.
The icon must be either \texttt{PNG}, \texttt{GIF} or \texttt{JPEG}
image format.

\textit{binarydigits} are 16, 64 or 256 binary digits (all 0's and 1's)
defining a square single color bitmap of size 4x4, 8x8 or 16x16 pixels.
Any other characters in the string are ignored.

\textit{hexdigits} are 4, 16 or 64 hexadecimal digits defining
a square single color bitmap image of size 4x4, 8x8 or 16x16 pixels.
Any non-hexadecimal digit characters in the string are ignored.

\textit{size} is the optional size for the icon in millimeters.  If no size
is given or size is zero then the icon is displayed at its natural size,
as it would appear in an image viewer with one image pixel per display
pixel.
The image is scaled and rotated according to the current \texttt{scale}
and \texttt{rotate} settings.

In PostScript files, icons with more than one color
are displayed with an opaque white background.  This is a limitation
of PostScript output.

Icon images are loaded into memory.
Loading very large images will use a large amount of memory.
Page \pageref{morememory} describes how to make more memory available
for Mapyrus.

\subsubsection{justify}

\begin{alltt}
justify \textit{justification}
\end{alltt}

Sets justification for labelling with the \texttt{label} command.
\textit{justification} is a string containing either
\texttt{left}, \texttt{right}, \texttt{center} for justifying labels
horizontally and/or
\texttt{top}, \texttt{middle}, \texttt{bottom} for justifying labels
vertically.

\subsubsection{key}

\begin{alltt}
key \textit{type}, \textit{description}, [\textit{arg1}, \textit{arg2} ...]
\end{alltt}

Defines an entry for a legend.  The procedure containing this command will be
called with arguments \textit{arg1}, \textit{arg2} ... to display a sample of
the symbol when a legend is generated with a \texttt{legend} command.  This
command is ignored if used outside of a procedure.

If \textit{description} contains the string \texttt{(\#)} then it will
be replaced in a legend by the number of times that the legend entry
is defined.

\textit{type} is either
\texttt{point} to display the legend entry as a single point,
\texttt{line} to display the legend entry as a horizontal line,
\texttt{zigzag} to display the legend entry as a zig-zag line,
or
\texttt{box} to display the legend as a box.
\textit{description} is the label for the legend entry.

If a procedure displays more than one type of type of symbol depending
on the arguments passed to it then use a separate
\texttt{key} command for each, with different descriptions
and different arguments.

\subsubsection{label}

\begin{alltt}
label \textit{string} [, \textit{string} ...]
\end{alltt}

Draws a label at each point in the path set with the \texttt{move} command,
using the font, justification and rotation set with the \texttt{font},
\texttt{justify} and \texttt{rotate} commands.  \textit{string} values are
separated by spaces.  If \textit{string} contains newline characters
(\texttt{\textbackslash{}n}) then labels are displayed as multiple lines, one
below the other, with each line justified independently.

\subsubsection{legend}

\begin{alltt}
legend \textit{size}
\end{alltt}

Displays legend entries defined with
\texttt{key} commands at points defined with
\texttt{move} commands.

Each legend entry corresponds to a procedure and a set of arguments.  The first
legend entry is displayed at the first \texttt{move} point by calling the
procedure in which the first legend entry was defined.  Then the second legend
entry is displayed at the second \texttt{move} point.  This continues until
either all legend entries are displayed or all move points are used.

The variable
\texttt{Mapyrus.key.count}
contains the number of legend entries that remain to be displayed.

If there are more legend entries than 
\texttt{move} points then some legend entries
remain undisplayed and will be displayed in the next legend.

Legend entries are displayed in the order in which they
are encountered in called procedures.

Legend entries in procedures that were never called are
not included in the legend.  Therefore, the legend only shows
entries that were actually displayed.

\textit{size} defines the size of each legend entry, in millimeters.

The description label is displayed to the right of each legend entry, using the
current \texttt{color}, \texttt{font} and \texttt{justify} settings.

\subsubsection{let}

\begin{alltt}
let \textit{var} = \textit{expression}, \dots
\end{alltt}

Assigns result of evaluating \textit{expression} to a variable with name
\textit{var}.  The variable is globally accessible unless defined as local to
the current procedure with a \texttt{local} command.

Variable \textit{var} is either a simple variable name, an array, or an array
element of the form \textit{var}[\textit{index}].

Several variables are assigned by separating each 
\textit{var} and \textit{expression} pair by a comma.

\subsubsection{linestyle}

\begin{alltt}
linestyle \textit{width}
linestyle \textit{width}, \textit{cap}, \textit{join}
linestyle \textit{width}, \textit{cap}, \textit{join}, \textit{phase}, \textit{dash length}, ...
\end{alltt}

Sets style line drawing by the \texttt{stroke}
command.
Line \textit{width} given in millimeters.
\textit{cap} is the style to use at the ends of lines, either
\texttt{butt}, \texttt{round} or \texttt{square}.
\textit{join} is the style to use where lines join, either
\texttt{bevel}, \texttt{miter} or \texttt{round}.
One or more \textit{dash length} values are given, alternating
between the length of one dash and the length of space between
dashes in a dash pattern.  Each \textit{dash length} is given in millimeters.
\textit{phase} is the offset in millimeters into the dash pattern 
at which to begin.

\subsubsection{local}

\begin{alltt}
local \textit{name}, [\textit{name} ...]
\end{alltt}

Declares the listed variable names as local to a procedure.
The variables are not visible outside the enclosing procedure
and their values are lost when the procedure ends.

\subsubsection{mimetype}

\begin{alltt}
mimetype \textit{type}
\end{alltt}

Sets MIME type for content being returned from HTTP request
when Mapyrus is running as an HTTP server.

\subsubsection{move}

\begin{alltt}
move \textit{x}, \textit{y}
\end{alltt}

Adds the point (\textit{x}, \textit{y}) to the current path.  The
point is first transformed through any world coordinate
transformation set with a \texttt{worlds} command,
then scaled and rotated by \texttt{scale}
and \texttt{rotate} values.

\subsubsection{newpage}

\begin{alltt}
newpage \textit{format}, \textit{filename}, \textit{width}, \textit{height}, \textit{extras}
\end{alltt}

Begins output of a new page to a file.  Any previous output is closed.  The
path, clipping path and world coordinates are cleared.  The origin of
the new page is in the lower-left corner.  \textit{format} is the
file format to use for output, one of:

\begin{itemize}
\item
\texttt{eps} for Encapsulated PostScript output,
with shapes and labels defined geometrically.
\item
\texttt{ps}, \texttt{postscript} or \texttt{application/postscript}
for PostScript output,
with shapes and labels defined geometrically.
\item
\texttt{screen} to display output in a window on the screen.
\item
\texttt{jpeg} or \texttt{image/jpeg} for JPEG image output.
\item
\texttt{png} or \texttt{image/png} for PNG image output.
\item
\texttt{epsimage} for Encapsulated PostScript image output.
Output contains
a single image in which all shapes and labels have been drawn.
\end{itemize}

See section \ref{pdf} for
information on converting PostScript output to PDF format.

\textit{width} and \textit{height} are the dimensions of the page
in millimeters.  To set dimensions in pixels for an image file,
use the calculation
\texttt{\textit{pixelSize} * Mapyrus.screen.dpi.mm} for
width and height.

\textit{filename} is the name of the file to write the page to.
If \textit{filename} is a dash (\texttt{-})
then the page is written to standard output.
If \textit{filename} begins with a pipe (\texttt{|}) then the rest
of \textit{filename} is interpreted as an operating system
command.  The operating system command is executed and Mapyrus
writes the page to the standard input of the executing
operating system command.
Mapyrus executes the operating command directly.  Set \textit{filename} to
\texttt{"| sh myscript.sh"}
or
\texttt{"| cmd myscript.bat"}
to execute the command in a sub-shell if 
environment variables, file redirection or other shell features are
required.

\textit{extras} defines further options for the new page, given as
\textit{variable=value} values, separated by whitespace.
See Table \ref{outputformats}
for options available for each type of output.

\begin{longtable}{|p{4cm}|p{10cm}|}
\hline
\label{outputformats}
File Format & Extras \\
\hline
\hline
\endfirsthead
\hline
\caption{Output Formats} \\
\endfoot

\hline
File Format & Extras \\
\hline
\hline
\endhead

PostScript, Encapsulated PostScript &

\texttt{background=\textit{color}}

Background color for page, as a named color or as hex digits.

\vspace{10pt}
\texttt{pfafiles=\textit{filename},\textit{filename2},...}

Comma-separated list of PostScript Type 1 font filenames
to include in this PostScript file.
A PostScript Type 1 font is defined in a file
with suffix \texttt{.pfa}.  Include
files for all PostScript fonts to be used in this
page that are not known by the printer.

See section \ref{psfonts} for
information on converting TrueType fonts to PostScript Type 1 format.

\vspace{10pt}
\texttt{isolatinfonts=\textit{fontname},\textit{fontname2},...}

Comma-separated list of PostScript Type 1 font names for
which ISO Latin1 character encoding
(also known as ISO-8859-1 encoding)
is required.  Use ISO Latin1 encoding
when extended characters such as accented characters
or a copyright symbol are to be displayed from the font.

\vspace{10pt}
\texttt{resolution=\textit{value}}

Resolution for page, given as a dots-per-inch value.  Replaces
default value of 300.

\vspace{10pt}
\texttt{turnpage=\textit{flag}}

If \textit{flag} is \texttt{true} then
turns a landscape orientation page 90 degrees so that it appears
as a portrait page.

\vspace{10pt}
\texttt{update=\textit{flag}}

If \textit{flag} is \texttt{true} then the file with name
\textit{filename} is an existing PostScript 
file that is opened for editing.  The existing file must be an
Encapsulated PostScript file or a PostScript file containing only
a single page.
Drawing commands will draw over the top of the existing page.
Page size is set to the size of the existing page,
\textit{width} and \textit{height} of the new page are ignored.  \\

\hline

PNG, JPEG images, output to a window on screen
and Encapsulated PostScript images &

\texttt{background=\textit{color}}

Background color for image, as a named color or as hex digits.

\vspace{10pt}
\texttt{labelantialiasing=\textit{flag}}

If \textit{flag} is \texttt{true} then
labels are drawn with anti-aliasing, improving readability.
Labels are drawn with anti-aliasing by default.

\vspace{10pt}
\texttt{lineantialiasing=\textit{flag}}

If \textit{flag} is \texttt{true} then
lines are drawn with anti-aliasing.
Lines are not drawn with anti-aliasing by default.

\vspace{10pt}
\texttt{resolution=\textit{value}}

Resolution for page, given as a dots-per-inch value.  Replaces
default value of 96.

\vspace{10pt}
\texttt{ttffiles=\textit{filename},\textit{filename2},...}

Comma-separated list of TrueType font filenames
to load for this page.
A TrueType font is defined in a file
with suffix \texttt{.ttf}.

Do not use this option on operating systems that support
TrueType fonts (Windows, Mac).  All TrueType
fonts are already available from the operating system.

On operating systems that do not support TrueType fonts
(Linux, UNIX) include filenames of all TrueType
fonts to be used on this page.  These fonts are loaded
by Mapyrus.

\vspace{10pt}
\texttt{update=\textit{flag}}

If \textit{flag} is \texttt{true} then the file with name
\textit{filename} is an existing file that is opened for editing.
Drawing commands will draw over the top of the existing image
in the file.
Page size is set to the size of the existing image,
\textit{width} and \textit{height} of the new page are ignored.  \\

\hline

\end{longtable}

\subsubsection{parallelpath}

\begin{alltt}
parallelpath \textit{distance} [, \textit{distance} ...]
\end{alltt}

Replaces current path with new paths parallel to current path.
For each given distance, a new path is created at \textit{distance}
millimeters to the right of current path.  If a \textit{distance} is
negative then path is created to the left of the current path.

When used on complex paths with sharp angles, this command creates
paths that self-intersect.

\subsubsection{print}

\begin{alltt}
print \textit{string} [, \textit{string} ...]
\end{alltt}

Prints each \textit{string} to standard output,
separated by spaces.
A newline is added after the final \textit{string}.

\subsubsection{protect}

\begin{alltt}
protect \textit{x1}, \textit{y1}, \textit{x2}, \textit{y2}
\end{alltt}

Marks a rectangular region of the page as protected.
The function \texttt{protected} will return 1 for any
point in this rectangular region.

The points
(\textit{x1}, \textit{y1}) and (\textit{x2}, \textit{y2}) define
any two opposite corners of the rectangle to mark as protected.

The four corner points of the rectangle
are first transformed through any world coordinate
transformation set with a \texttt{worlds} command,
then scaled and rotated by \texttt{scale}
and \texttt{rotate} values.

The rectangular region protected is always aligned with the X
and Y axes of the page, regardless of any rotation angle.

\subsubsection{rotate}

\begin{alltt}
rotate \textit{angle}
\end{alltt}

Rotates the coordinate system, adding to any existing rotation.  \textit{angle}
is given in degrees, measured counter-clockwise.  All later coordinates given
in \texttt{move}, \texttt{draw}, \texttt{arc} and \texttt{addpath} commands are
rotated by this angle.

\subsubsection{rdraw}

\begin{alltt}
rdraw \textit{dx}, \textit{dy}, ...
\end{alltt}

Adds one or more straight line segments to the current path
using relative distances.
A straight line segment is added from the previously defined point
a relative distance (\textit{dx}, \textit{dy}).  Each further
point adds a line segment relative to the point before.
Points are first transformed through any world coordinate
transformation set with a \texttt{worlds} command
then scaled and rotated by \texttt{scale}
and \texttt{rotate} values.

\subsubsection{roundedbox}

\begin{alltt}
roundedbox \textit{x1}, \textit{y1}, \textit{x2}, \textit{y2}
roundedbox \textit{x1}, \textit{y1}, \textit{x2}, \textit{y2}, \textit{radius}
\end{alltt}

Adds a rectangle with rounded corners to the current path.
The points
(\textit{x1}, \textit{y1}) and (\textit{x2}, \textit{y2}) define
any two opposite corners of the rectangle.

The radius of circular arcs at the rounded corners is 
\textit{radius}, or 10\% of the size of the rectangle if not given.

The points and radius are transformed through any
transformation set with a \textit{worlds} command,
then scaled and rotated by \textit{scale}
and \textit{rotate} values.

\subsubsection{samplepath}

\begin{alltt}
samplepath \textit{spacing}, \textit{offset}
\end{alltt}

Replaces current path with equally spaced points along the path.
\textit{offset} is the distance along the path at which to place first point,
given in millimeters.  \textit{spacing} is the distance between points, given
in millimeters.  The sign of \textit{spacing} controls the direction in which
the path is travelled.  If \textit{spacing} is a positive value, the path is
travelled from the beginning towards the end.  If \textit{spacing} is a
negative value, then the absolute value of \textit{spacing} is used and the
path is travelled from the end towards the beginning.  Using a very large
postive or negative value for \textit{spacing} results in current path being
replaced by a single point at the beginning or end of the path.

\subsubsection{scale}

\begin{alltt}
scale \textit{factor}
\end{alltt}

Scales the coordinate system, adding to any existing scaling.  \textit{factor}
is scale factor for X and Y axes.  All later coordinates given in
\texttt{move}, \texttt{draw}, \texttt{arc} and \texttt{addpath} commands are
scaled by this factor.

\subsubsection{shiftpath}

\begin{alltt}
shiftpath \textit{x}, \textit{y}
\end{alltt}

Shifts all points in the current path \textit{x} millimeters along the X axis
and \textit{y} millimeters along the Y axis.  Shift values are scaled and
rotated by \texttt{scale} and \texttt{rotate} values but are independent
of any world coordinate transformation.

Use this command
repeatedly following a \texttt{clip "outside"} command to produce a shadow
effect for polygons, as shown in 
Section \ref{tutorialshadow} on page \pageref{tutorialshadow}.

\subsubsection{sinkhole}

\begin{alltt}
sinkhole
\end{alltt}

Replaces the current path containing a polygon with a single point in the
middle of the polygon, farthest from the polygon perimeter.

\subsubsection{star}

\begin{alltt}
star \textit{xCenter}, \textit{yCenter}, \textit{radius}, \textit{points}
\end{alltt}

Adds a star shape to the current path, with center
point (\textit{xCenter}, \textit{yCenter}), distance
\textit{radius}
from the center to each point of the star.
\textit{points} is the number of points for the star.

The center point and radius are transformed through any
transformation set with a \textit{worlds} command,
then scaled and rotated by \textit{scale}
and \textit{rotate} values.

\subsubsection{stripepath}

\begin{alltt}
stripepath \textit{spacing}, \textit{angle}
\end{alltt}

Replaces current path with equally spaced parallel lines that completely cover
the path.  \textit{spacing} is the distance between lines, measured in
millimeters.  \textit{angle} is angle of each line, measured
counter-clockwise in degrees, with zero being horizontal.  Follow this command
with a \texttt{clip} command to produce a hatched fill pattern.

\subsubsection{stroke}

\begin{alltt}
stroke
\end{alltt}

Draws the current path using the current color and linestyle.
The current path is not.

\subsubsection{triangle}

\begin{alltt}
triangle \textit{xCenter}, \textit{yCenter}, \textit{radius}, \textit{rotation}
\end{alltt}

Adds an equilateral triangle to the current path, with center
point (\textit{xCenter}, \textit{yCenter}) and distance
\textit{radius}
from the center point to each vertex.

The triangle is rotated clockwise \textit{rotation} degrees.

The center point, radius and rotation are transformed through any
transformation set with a \textit{worlds} command,
then scaled and rotated by \textit{scale}
and \textit{rotate} values.

\subsubsection{unprotect}

\begin{alltt}
unprotect \textit{x1}, \textit{y1}, \textit{x2}, \textit{y2}
\end{alltt}

Clears all protected regions from the page in a rectangular
area.

The points
(\textit{x1}, \textit{y1}) and (\textit{x2}, \textit{y2}) define
any two opposite corners of the rectangle to clear.

The four corner points of the rectangle
are first transformed through any world coordinate
transformation set with a \texttt{worlds} command,
then scaled and rotated by \texttt{scale}
and \texttt{rotate} values.

The rectangular region cleared is always aligned with the X
and Y axes of the page, regardless of any rotation angle.

\subsubsection{wedge}

\begin{alltt}
wedge \textit{xCenter}, \textit{yCenter}, \textit{radius}, \textit{angle}, \textit{sweep}
\end{alltt}

Adds a wedge (pie slice) shape to the current path, with center
point (\textit{xCenter}, \textit{yCenter}) and radius
\textit{radius}.
The wedge begins at angle \textit{angle} measured
counter-clockwise in degrees, with zero being horizontal.
The wedge is open \textit{sweep} degrees in a counter-clockwise direction.
If \textit{sweep} is negative then the wedge opens in a clockwise direction.

The center point and radius are transformed through any
transformation set with a \textit{worlds} command,
then scaled and rotated by \textit{scale}
and \textit{rotate} values.

\subsubsection{worlds}

\begin{alltt}
worlds \textit{wx1}, \textit{wy1}, \textit{wx2}, \textit{wy2} [, \textit{extras} ]
worlds \textit{wx1}, \textit{wy1}, \textit{wx2}, \textit{wy2}, \textit{px1}, \textit{py1}, \textit{px2}, \textit{py2} [, \textit{extras} ]
\end{alltt}

Defines a world coordinate system for the page.

The coordinates
(\textit{wx1}, \textit{wy1}) 
and
(\textit{wx2}, \textit{wy2})
define the lower-left and upper-right world coordinate values.

The coordinates
(\textit{px1}, \textit{py1}) 
and
(\textit{px2}, \textit{py2})
define the lower-left and upper-right positions on the page
in millimetres.  The world coordinates are mapped into this area
of the page.  If page coordinates are not given then the world
coordinates are mapped to the whole page.

The new world coordinates replace any world coordinates
set with a previous \texttt{worlds} command.

\textit{extras} defines further options, given as
\textit{variable=value} values, separated by whitespace.
See Table \ref{worldsextras}
for available options.

\begin{longtable}{|l|p{7cm}|}
\hline
\label{worldsextras}
Extra & Description \\
\hline
\hline
\endfirsthead
\hline
\caption{Font Extras} \\
\endfoot

\hline
Extra & Description \\
\hline
\hline
\endhead

\texttt{units=\textit{units}} &

Defines the units of the world coordinates,
either \texttt{metres}, \texttt{meters} or \texttt{feet}.
If not given, units are assumed to be meters. \\

\texttt{distortion=\textit{flag}} &

If \texttt{true} then non-uniform scaling in X and Y axes
is allowed.  If \texttt{false} then
the world coordinate range is expanded,
if necessary, to maintain uniform scaling.
If not given, then scaling is uniform. \\

\hline
\end{longtable}

\subsection{Error Handling}

If Mapyrus encounters an error when interpreting commands,
an error message is printed including the filename and line number
at which the error occurred and Mapyrus exits immediately.
The Java interpreter exits with a non-zero status.

If an error occurs when using the HTTP Server
option, an HTTP failure status and the error message
are returned to the HTTP client.  The HTTP server continues,
handling the next request.

\subsection{Mapyrus HTTP Server}

Mapyrus runs as an HTTP server when started with the \texttt{-s} command line
option.

The HTTP server accepts and replies to requests from HTTP clients on the given
port number.  If port number is 0 then any free port number is used.  The port
number used is written to the log file or to standard output.

The HTTP server is multi-threaded to enable several requests to be handled
simultaneously.  If the HTTP server receives an HTTP request for a file with a
suffix matching a well-known MIME type (\texttt{html}, \texttt{txt},
\texttt{ps}, \texttt{pdf}, \texttt{zip} or a web image format), then the
contents of that file are returned by the HTTP server to the HTTP client.
Requests for files with no suffix, or with unknown file suffix
such as \texttt{.mapyrus} are interpreted
by Mapyrus using the following steps.

\begin{enumerate}
\item
Set any parameters passed in the URL (following the \texttt{?} character in
the URL or passed in an HTML form) as variables in Mapyrus.
Uppercase variable names are used.
\item
Read and execute the commands from the filename given in the URL.
\item
Capture the standard output of these commands and return it to the
HTTP client.
An image file is returned if the
\texttt{newpage} command is used with output file set
to standard output.
Otherwise the output of any \texttt{print} commands is returned.
The MIME type set in a \texttt{mimetype} command is returned as
the content type to the HTTP client.

An HTTP error state is returned if the request fails.
\end{enumerate}

Requests using either GET or POST methods are accepted by Mapyrus.

The HTTP server runs forever and
is stateless.  Each HTTP request is independent and
variables, graphics state and legend entries are not shared between
requests.  Cookies are not used.
Any files or URLs given on the command line when Mapyrus is started
are interpreted before accepting HTTP requests.
Procedures defined in these files
are available when interpreting HTTP requests.  This enables
common procedures to be loaded only once at startup and not with every
HTTP request.

Files are not cached and are read for each HTTP request.

For security, the HTTP server only replies to requests
from the directory in which Mapyrus was started.
Requests for files from other directories return an error to the HTTP client.
If communication between HTTP client and Mapyrus is blocked for longer than
5 minutes then the HTTP request is cancelled.
When all threads in the HTTP server are busy handling requests,
further requests are queued.

Logging of HTTP requests is controlled by setting a startup variable (see Table
\ref{startupvariables} on page \pageref{startupvariables}) defining a logging
configuration file.

