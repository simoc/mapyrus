% @(#) $Id$

\section{Introduction}

Mapyrus is software for creating plots of points, lines, polygons and labels 
to PostScript (high resolution, up to A0 paper size),
Portable Document Format (PDF) and web image output formats.

The software combines the following three components.

\begin{enumerate}
\item

A Logo or turtle graphics program.  An imaginary pen is moved around a page,
creating shapes that are drawn into an image file.
Reusable routines are built up using a BASIC-like language.
Branching and looping constructs enable complex shapes, symbols and patterns
to be be defined.  See Figure \ref{turtle}.

\begin{figure}[h]
\includegraphics{turtle1.eps}
\includegraphics{turtle2.eps}
\includegraphics{turtle3.eps}
\includegraphics{turtle4.eps}
\includegraphics{turtle5.eps}
\includegraphics{turtle6.eps}
\caption{Shapes, Symbols And Patterns}
\label{turtle}
\end{figure}

\item

Reading and displaying of geographic information
systems (GIS) datasets, text files, or tables held in a relational database
(including spatial databases such as PostGIS and Oracle Spatial).
Drawing routines are applied to geographic data to produce annotated and
symbolized maps and graphs.  Attributes of the geographic data control
the color, size, annotation and other characteristics of the
appearance of the geographic data.
Scalebars, legends, coordinate grids and north arrows are also available.
See Figures \ref{mapview1}, \ref{mapview3}, \ref{mapview2} and
\ref{mapview4}.

\begin{figure}
\includegraphics{mapview1.eps}
\caption[Average Monthly Temperatures]{Average Monthly Temperatures of Australian Cities (degrees Celsius)}
\label{mapview1}
\end{figure}

\begin{figure}
\includegraphics{mapview3.eps}
\caption{Railways Lines in East Kent}
\label{mapview3}
\end{figure}

\begin{figure}
\includegraphics{mapview2.eps}
\includegraphics{mapview2legend.eps}
\caption{Vegetation Classes}
\label{mapview2}
\end{figure}

\begin{figure}
\includegraphics{mapview4.eps}
\caption{Inventory Levels at Warehouses}
\label{mapview4}
\end{figure}

\item
Mapyrus runs as either
a stand-alone program for integration into
scripts and batch tasks  (this is suitable for generating a one-off
map or a series of similar maps from a template,
showing different areas, or using different criteria for each map)
or as a self-contained web server providing HTML pages, map and
graph images and text reports to a web-based application.

\end{enumerate}

Please note that:

\begin{itemize}
\item
Mapyrus is written in Java so as to be operating system independent
and take advantage of Java multi-threading, security,
Java2D and JDBC capabilities.

\item
Mapyrus has no graphical user interface.
Commands are typed into the terminal window in which the software is running,
or read from a text file prepared by the user.

\item
Mapyrus cannot display three dimensional data and has nothing to do
with 3D.

\item
Mapyrus does not include a library of ready-to-use symbols and patterns.
Users must define their own, using examples in this manual as a basis.
However, a large number of free TrueType format fonts found on the internet
defining shapes for animals, arrows and pointers, road signs, etc. are
available for use in Mapyrus.

\end{itemize}

